% FortySecondsCV LaTeX template
% Copyright © 2019-2020 René Wirnata
% Licensed under the 3-Clause BSD License.
%
% Fuente: https://github.com/PandaScience/FortySecondsCV

%-------------------------------------------------------------------------------
%                             ADDITIONAL PACKAGES
%-------------------------------------------------------------------------------
\documentclass[
    a4paper,
    % showframes,
    % vline=2.2em,
    % maincolor=cvblack,
    % sidecolor=gray!50,
    % sectioncolor=cvblack,
    subsectioncolor=cvblue,
    % itemtextcolor=black!80,
    % sidebarwidth=0.4\paperwidth,
    % topbottommargin=0.03\paperheight,
    % leftrightmargin=20pt,
    % profilepicsize=4.5cm,
    profilepicborderwidth=4.5pt,
    % profilepicstyle=profilecircle,
    % profilepiczoom=1.0,
    % profilepicxshift=0mm,
    % profilepicyshift=0mm,
    % profilepicrounding=1.0cm,
]{fortysecondscv}
% improve word spacing and hyphenation
\usepackage{microtype}
\usepackage{ragged2e}
\usepackage[spanish]{babel}

% Desactivar la separación de palabras con guión
\usepackage[none]{hyphenat} %% CORRECCIÓN: Se desactiva la división automática de palabras.

% take care of proper font encoding
\ifxetexorluatex
    \usepackage{fontspec}
    \defaultfontfeatures{Ligatures=TeX}
%   \newfontfamily\headingfont[Path = fonts/]{segoeuib.ttf} % local font
\else
    \usepackage[utf8]{inputenc}
    \usepackage[T1]{fontenc}
%   \usepackage[sfdefault]{noto} % use noto google font
\fi

% enable mathematical syntax for some symbols like \varnothing
\usepackage{amssymb}

% bubble diagram configuration
\usepackage{smartdiagram}
\smartdiagramset{
    bubble center node font = \footnotesize,
    bubble node font = \footnotesize,
    bubble center node size = 0.4\sidebartextwidth,
    bubble node size = 0.25\sidebartextwidth,
    distance center/other bubbles = 1.5em,
    bubble center node color = maincolor!70,
    set color list = {maincolor!10, maincolor!40, maincolor!20, maincolor!60, maincolor!35},
    bubble fill opacity = 0.8,
}


%-------------------------------------------------------------------------------
%                            PERSONAL INFORMATION
%-------------------------------------------------------------------------------
%% mandatory information
\cvname{\LARGE\textbf{Jose Daniel Garcia Arias}}
\cvjobtitle{Ingeniero Electrónico\\[0.2em]MSc en Ingeniería de Software} %% CORRECCIÓN: Unificación y correcta capitalización.
\cvprofilepic{pics/profile.jpg}
\cvbirthday{Julio 11, 1996}
\cvphone{+57 317 863 1915}
% \cvmail{jd.garciaa1@uniandes.edu.co}
\cvsite{www.josegarcia.com.co}
\cvmail{jose.garcia960711@gmail.com}
\cvcustomdata{\faFlag}{Colombiano}
\cvcustomdata{\faIdCard}{Tarjeta Profesional}
\cvcustomdata{\faCheck}{Disponibilidad para viajes y traslado}

%-------------------------------------------------------------------------------
%                              SIDEBAR 1st PAGE
%-------------------------------------------------------------------------------
\addtofrontsidebar{
    \graphicspath{{pics/flags/}}
    \profilesection{Redes Profesionales}
        \begin{icontable}{2.5em}{1em}
            \social{\faLinkedin}{https://www.linkedin.com/in/jgarciaarias/}{Linkedin (clic)}
            \social{\faGithub}{https://github.com/jdanielkc}{Github Link}
        \end{icontable}
    \profilesection{Lenguajes}
        \pointskill{\flag{CO.png}}{Español}{5}
        \pointskill{\flag{GB.png}}{Inglés}{3}
    \profilesection{Habilidades Duras}
            \pointskill{\faJava}{Java}{5}[5]
            \pointskill{\faGitSquare}{Git y GitHub}{5}[5] %% CORRECCIÓN: Capitalización unificada.
            \pointskill{\faDatabase}{SQL}{5}[5]
            \pointskill{\faJava}{Spring Framework}{4}[5]
            \pointskill{\faJava}{JMeter}{4}[5] %% CORRECCIÓN: "JMeter" con M mayúscula.
            \pointskill{\faAngular}{Angular}{4}[5]
            \pointskill{\faDatabase}{PL/SQL}{4}[5]
            \pointskill{\faHtml5}{HTML}{4}[5]
            \pointskill{\faCss3*}{CSS}{4}[5]
            \pointskill{\faPython}{Python}{4}[5]
            \pointskill{\faDocker}{Docker}{4}[5]
            \pointskill{\faMaxcdn}{Maven}{4}[5]
            \pointskill{\faAws}{AWS}{3}[5]
            \pointskill{\faJsSquare}{JavaScript}{3}[5]
}

%-------------------------------------------------------------------------------
%                              SIDEBAR 2nd PAGE
%-------------------------------------------------------------------------------
\addtobacksidebar{
    \profilesection{Sobre mí}
    \aboutme{
    \textcolor{black!90}{Soy un ingeniero electrónico con experiencia en desarrollo de software full stack y arquitectura cloud. He participado en la integración de importantes empresas del país, como Grupo Éxito, Supergiros y Carvajal Tecnología y Servicios. Mis fortalezas técnicas incluyen Java, Spring Boot, Angular, AWS, PL/SQL y arquitecturas modernas como Hexagonal y Orientada a Eventos.\\
    Actualmente curso el último semestre de mi Maestría en Ingeniería de Software en la Universidad de los Andes, profundizando mis conocimientos en arquitectura de software, desarrollo web, pruebas automatizadas y aplicaciones en la nube.\\
    A lo largo de mi carrera he trabajado traduciendo requisitos del negocio en soluciones técnicas efectivas, aplicando principios SOLID y Clean Code, optimizando recursos cloud y abordando desafíos complejos en distintos dominios del desarrollo de software. Mi formación multidisciplinaria me permite adaptarme a diversos contextos tecnológicos y contribuir en diferentes áreas de la ingeniería de software.}}
    
    \profilesection{Habilidades Blandas}
        \skill{\faUsers}{Trabajo en equipo}{}
        \skill{\faLightbulb[regular]}{Curiosidad}{}
        \skill{\faHandPaper[regular]}{Iniciativa}{}
        \skill{\faCogs}{Adaptabilidad}{}
        \skill{\faClock}{Gestión del tiempo}{}
        \skill{\faComments}{Comunicación}{}
        \skill{\faPuzzlePiece}{Resolución de Problemas}{}
        \skill{\faUserTie}{Liderazgo}{}

        
    \profilesection{Software}
    \begin{sidebarminipage}
            \chartlabel{Azure}
            \chartlabel{Microsoft Office}
            \chartlabel{Power BI}
            \chartlabel{Postman}
            \chartlabel{Postgres DB}
            \chartlabel{Libre Office}
            \chartlabel{LaTeX}
            \chartlabel{Keras}
            \chartlabel{Oracle DB}
            \chartlabel{JMeter} %% CORRECCIÓN: Consistencia con "JMeter".
            \chartlabel{SO Linux}
            \chartlabel{SO Windows}
            \chartlabel{Pandas}
    \end{sidebarminipage}
}

%-------------------------------------------------------------------------------
%                         TABLE ENTRIES RIGHT COLUMN
%-------------------------------------------------------------------------------
% Definición de un item flexible (se mantiene por si deseas usarlo en el futuro)
\newcommand{\cvitemfluid}[4]{
    \par\noindent
    \begin{minipage}[t]{0.25\textwidth}
        \raggedright #1
    \end{minipage}%
    \begin{minipage}[t]{0.74\textwidth}
        \textbf{#2} \hfill {\footnotesize #3}
    \end{minipage}\\[0.2em]
    \begingroup
    \leftskip=0.25\textwidth
    \noindent #4
    \par
    \endgroup
    \vspace{1em}
}

\begin{document}

\makefrontsidebar

\cvsection{Experiencia Laboral}

% --- BLOQUE 1: PÁGINA 1 (Trabajos Recientes) ---
\begin{cvtable}[3]
    \cvitem{05/2025 -- Actualmente\centering}{Desarrollador Senior Profesional}{Carvajal Tecnología y Servicios}{
        Como parte de un equipo de desarrollo de alto nivel, participo en la construcción y evolución de un sistema de gestión documental regional (Colombia, México y Perú). Mi rol se enfoca en la implementación técnica de soluciones full stack, la optimización de recursos en la nube y la aplicación de estándares de calidad de software.\\[0.2em]
        • Implementación de estrategias de almacenamiento eficiente en AWS (S3 Deep Archive, Glacier), logrando una reducción de costos operativa mediante la gestión inteligente del ciclo de vida de la información.\\[0.2em]
        • Desarrollo de componentes bajo arquitectura Hexagonal y Orientada a Eventos, desplegados en servicios de contenedores (EKS, ECS) y gestionados mediante servicios de identidad (AWS Cognito, IAM).\\[0.2em]
        • Aplicación de principios SOLID, Clean Code y patrones de diseño en un entorno de metodologías ágiles, asegurando la mantenibilidad y escalabilidad del código entregado.\\[0.2em]
        \textbf{Skills}: Java, Spring Boot, Angular, AWS (S3, EKS, ECS, IAM, Cognito), Arquitectura Hexagonal, Event-Driven Architecture, TypeScript, Scrum, HTML, CSS.
    }
    \cvitem{10/2022 -- 05/2025\centering}{Ingeniero de Desarrollo}{Red Empresarial de Servicios S.A}{
        Desarrollador FullStack en integraciones de gran escala (Supergiros, Grupo Éxito) bajo Scrum. Liderazgo técnico aplicando principios SOLID y patrones de arquitectura (MVC, SAGA) en soluciones web y móviles.\\[0.2em]
        • Desarrollo seguro en Java LTS (8, 11, 17) implementando protocolos TLS/SSL y mTLS.\\[0.2em]
        • Diseño y optimización de bases de datos (Postgres, Oracle) asegurando integridad y eficiencia.\\[0.2em]
        • Integración de sistemas vía API REST y migración móvil a stack moderno (Angular, Spring, Flutter).\\[0.2em]
        • Ejecución de pruebas de carga y unitarias (JMeter, JUnit, Mockito, Spring Test).\\[0.2em]
        \textbf{Skills}: Java Spring Boot, Angular, PL/SQL, Oracle/Postgres, Docker, Azure, Microservicios, JMeter, Git/SVN/Flow, CI/CD, HTML/CSS/JS.
    }
    \cvitem{01/2022--07/2022\centering}{Analytic and Big Data Intern}{Accenture Colombia}{
        Desempeñé el rol de ingeniero de datos junior, colaborando en el desarrollo, implementación y entrega de un modelo de aprendizaje automático para predecir la probabilidad de abandono de compra. Entre mis responsabilidades se destacó la optimización de modelos de inteligencia artificial mediante ajuste de hiperparámetros y la generación de dashboards en PowerBI para visualizar los resultados.\\[0.2em]
        • Optimización de modelos utilizando técnicas como Support Vector Machine, Random Forest y Deep Networks.\\[0.2em]
        • Desarrollo de bases de datos optimizadas para el entrenamiento de modelos de Machine Learning.\\[0.2em]
        • Diseño de dashboards en PowerBI para el monitoreo de resultados.\\[0.2em]
        \textbf{Skills}: Python, NumPy, Inteligencia Artificial, Apache Spark, Pandas, PowerBI, Machine Learning, Keras, TensorFlow.
    }
\end{cvtable}

% --- CORTE DE PÁGINA Y ACTIVACIÓN DE BARRA LATERAL ---
\newpage
\makebacksidebar

% --- BLOQUE 2: PÁGINA 2 (Trabajos Anteriores) ---
\begin{cvtable}[3]

    \cvitem{01/2019--12/2021\centering}{Tecnólogo de Informática y Soporte}{Diverjuegos SAS}{
        Responsable del diseño, desarrollo e implementación de requerimientos de software, así como del soporte técnico de la infraestructura tecnológica y la supervisión de los sistemas de seguridad informática.\\[0.2em]
        • Diseño de la arquitectura de microservicios para un portal administrativo tipo wiki, destinado a documentar las capacitaciones de nuevos empleados.\\[0.2em]
        • Desarrollo del BackEnd y FrontEnd utilizando Spring Data, Spring Boot, Spring Gateway y Angular.\\[0.2em]
        % • Creación de una aplicación web SPA con Bootstrap, HTML, CSS y Angular.\\[0.2em]
        • Mantenimiento y soporte técnico en las distintas sedes de la empresa.\\[0.2em]
        \textbf{Skills}: Spring Framework, Soporte Técnico, Java, Docker, Spring Boot, Angular, Seguridad Informática, Diseño de Sistemas IP, HTML, CSS, JavaScript.
    }
\end{cvtable}

\cvsection{Estudios}
\cvsubsection{\textbf{Profesionales}}
\begin{cvtable}[1.5]
    \cvitem{2024--Actualmente\centering}{Maestría en Ingeniería de Software}{Universidad de los Andes}{Actualmente curso en último semestre de la maestría, abarcando temas como ingeniería y arquitectura de software web, pruebas automatizadas, aplicaciones web, aplicaciones en la nube, entre otras. }
    \cvitem{2015 -- 2023\centering}{Ingeniería Electrónica}{Universidad del valle}{\textit{Trabajo de grado}: Clasificación de patologías cardiovasculares con máquinas de aprendizaje profundo a partir de señales ECG.}
    \cvitem{2021 -- 2021\centering}{Diplomado en Programación web}{U. Tecnológica de Pereira}{El diplomado abarcó 4 áreas: Python, JAVA, desarrollo de software y aplicaciones web a lo largo de 800 horas.}
\end{cvtable}

\cvsubsection{\textbf{Certificaciones}}
\begin{cvtable}[1.5]
    \cvitem{2023\centering}{Scrum Foundation Professional Certificate SFPC}{CertiProf}{}
    \cvitem{2023\centering}{Scrum Fundamentals Certified (SFC)}{Scrumstudy}{}
\end{cvtable}

\cvsubsection{\textbf{Cursos}}
\begin{cvtable}[1.5]
    \cvitem{2024\centering}{Control de versiones con Git y GitHub}{Universidad de los Andes}{}
    \cvitem{2024\centering}{Introducción a UML}{Universidad de los Andes}{}
\end{cvtable}
\begin{cvtable}[1.5]
    \cvitem{2024\centering}{Programación en Python}{Universidad de los Andes}{}
    \cvitem{2023\centering}{PL/SQL Desde Cero}{Udemy}{}
    % \cvitem{2022\centering}{Curso profesional de Git y Github}{Platzi}{}
    \cvitem{2022\centering}{Python for everybody (specialization)}{University of Michigan}{}
    % \cvitem{2022\centering}{Using Python to Access Web Data}{University of Michigan}{}
    % \cvitem{2021\centering}{Python Data Structures}{University of Michigan}{}
    % \cvitem{2021\centering}{Getting Started with Python}{University of Michigan}{}
\end{cvtable}

\cvsection{Proyectos}
\begin{cvtable}
    \cvitem{\centering2022 -- 2023}{Ingeniería Electrónica}{Universidad del valle}{Diseño e implementación de un sistema de clasificación de patologías cardiovasculares haciendo uso de redes neuronales recurrentes y convolucionales a partir de señales electrocardiográficas.}
\end{cvtable}

\cvsection{Reconocimientos}
\begin{cvtable}
    \cvitem{08/2020-12/2020\centering}{Reconocimiento por el mejor promedio académico del semestre.}{Universidad del Valle}{}\\
    \cvitem{10/2021-04/2022\centering}{Reconocimiento por el mejor promedio académico del semestre.}{Universidad del Valle}{}
\end{cvtable}

% \cvsection{Referencias Personales}
% \begin{cvtable}
%     \cvitem{}{Diego Fernando Teuta}{Cenicaña}{Cargo: Ingeniero Electrónico\\ Contacto: +57 315 7404753}\\
%     \cvitem{}{Angela Andrade}{Kalettre}{Cargo: Desarrolladora de software\\ Contacto: +57 318 6875106}\\
% \end{cvtable}

% \cvsignature
\end{document}
